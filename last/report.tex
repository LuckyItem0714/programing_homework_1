\documentclass[12pt,a4paper]{jsarticle}
\usepackage[utf8]{inputenc}
\usepackage[japanese]{babel}
\usepackage{amsmath}
\usepackage{amsfonts}
\usepackage{amssymb}
\usepackage{graphicx}
\usepackage{url}
\usepackage{listings}
\usepackage{xcolor}
\usepackage{geometry}
\usepackage{fancyhdr}
\usepackage{booktabs}
\usepackage{array}
\usepackage{longtable}

% ページ設定
\geometry{left=25mm,right=25mm,top=30mm,bottom=30mm}

% ヘッダー・フッター設定
\pagestyle{fancy}
\fancyhf{}
\fancyhead[L]{システム統合レポート}
\fancyhead[R]{\thepage}
\renewcommand{\headrulewidth}{0.4pt}

% タイトル情報
\title{プログラミング応用 最終課題レポート\\
\large 新システム提案・計画書}
\author{
グループ名:グループ13 フル単\\
メンバー:井田 礼慈(学籍番号:35714012)\\
\quad\quad\quad 大橋 蒼一朗(学籍番号:35714026)\\
\quad\quad\quad 松岡 遼(学籍番号:35714128)\\
}
\date{\today}

\begin{document}

\maketitle

\newpage

\section{システムの概要}
手話を自然な日本語で読み上げるシステム。手話の映像からAIによって意味を識別し、自然な日本語に変換する。
\section{背景}

% 現在のシステムやプロセスにおける問題点を記述
聴覚障害者の主なコミュニケーション手段として手話と筆談があるが、手話の習得は難しく、手話が通じる人の数は少ない現状がある。


\section{目的}

% プロジェクトの全体的な目標を記述
聴覚障害者のコミュニケーションの障壁をなくし、聞き手に手話の知識がなくても、会話ができるようにすることを目的とする。

\section{実現上の課題}

\begin{itemize}
    \item 映像から単語への変換
    \item 単語列から自然な文章への変換
    \item 精度
\end{itemize}

\section{解決法}

\subsection{映像から単語への変換}
% 技術的課題に対する解決策を詳述
既存のLLMをもとに転移学習によって手話の映像から単語に変換するAIを作成する。
\subsubsection{単語列から自然な文章への変換}
% 提案するシステムのアーキテクチャを説明
既存のLLMを利用する。
\subsubsection{精度}
% データ統合の具体的手法を説明
画像認識に加え、専用のモーションキャプチャを行う手袋を作成し、利用することで精度を向上させる。\\
翻訳ミスをデータベースに記録し、一定期間の後、再度チューニングを行うことで地域差や個人差に適応する。\\
\subsubsection{セキュリティ対策}
% セキュリティに関する解決策を説明

\subsection{運用上の解決策}
% 運用面での解決策を詳述

\subsubsection{移行戦略}
% システム移行の戦略を説明

\subsubsection{ユーザーサポート体制}
% ユーザーサポートの体制を説明

\subsection{組織的解決策}
% 組織・人的課題に対する解決策を詳述

\section{実装工程表}

\subsection{プロジェクト全体スケジュール}
% 全体のスケジュールを記述

\begin{table}[h]
\centering
\caption{実装工程表}
\begin{tabular}{|l|l|l|l|}
\hline
\textbf{フェーズ} & \textbf{期間} & \textbf{主要作業} & \textbf{成果物} \\
\hline
要件定義 & 20XX年X月~X月 & 要件の詳細化、仕様策定 & 要件定義書 \\
\hline
基本設計 & 20XX年X月~X月 & システム設計、DB設計 & 基本設計書 \\
\hline
詳細設計 & 20XX年X月~X月 & 詳細仕様策定 & 詳細設計書 \\
\hline
開発・実装 & 20XX年X月~X月 & プログラミング、単体テスト & システム \\
\hline
統合テスト & 20XX年X月~X月 & 統合テスト、性能テスト & テスト報告書 \\
\hline
移行・導入 & 20XX年X月~X月 & データ移行、本稼働 & 運用開始 \\
\hline
\end{tabular}
\end{table}

\subsection{マイルストーン}
% 重要なマイルストーンを記述
\begin{enumerate}
    \item 要件定義完了:20XX年X月X日
    \item 基本設計完了:20XX年X月X日
    \item 開発完了:20XX年X月X日
    \item 本稼働開始:20XX年X月X日
\end{enumerate}

\subsection{リスク管理}
% プロジェクトリスクと対策を記述

\section{効果}

\subsection{定量的効果}
% 数値で測定可能な効果を記述
\begin{table}[h]
\centering
\caption{定量的効果の試算}
\begin{tabular}{|l|r|r|r|}
\hline
\textbf{項目} & \textbf{現状} & \textbf{改善後} & \textbf{改善率} \\
\hline
処理時間 & 60分 & 15分 & 75\%削減 \\
\hline
エラー率 & 5\% & 1\% & 80\%削減 \\
\hline
運用コスト & 100万円/月 & 60万円/月 & 40\%削減 \\
\hline
\end{tabular}
\end{table}

\subsection{定性的効果}
% 数値化困難だが重要な効果を記述
\begin{itemize}
    \item ユーザビリティの向上
    \item 業務効率の改善
    \item データ品質の向上
    \item 意思決定の迅速化
\end{itemize}

\subsection{投資対効果(ROI)}
% 投資対効果の分析結果を記述

\section{ポンチ絵(システム概要図)}

\subsection{現行システム構成}
% 現行システムの構成図を挿入
\begin{figure}[h]
\centering
% \includegraphics[width=0.8\textwidth]{current_system.png}
\fbox{\parbox{0.8\textwidth}{\centering 現行システム構成図\\(図を挿入してください)}}
\caption{現行システム構成}
\end{figure}

\subsection{提案システム構成}
% 提案するシステムの構成図を挿入
\begin{figure}[h]
\centering
% \includegraphics[width=0.8\textwidth]{proposed_system.png}
\fbox{\parbox{0.8\textwidth}{\centering 提案システム構成図\\(図を挿入してください)}}
\caption{提案システム構成}
\end{figure}

\subsection{統合アーキテクチャ概要}
% システム統合のアーキテクチャ概要図を挿入
\begin{figure}[h]
\centering
% \includegraphics[width=0.8\textwidth]{integration_architecture.png}
\fbox{\parbox{0.8\textwidth}{\centering 統合アーキテクチャ概要図\\(図を挿入してください)}}
\caption{統合アーキテクチャ概要}
\end{figure}

\section{まとめ}

\subsection{提案の要点}
% 提案内容の要点をまとめ

\subsection{今後の課題}
% 今後検討すべき課題を記述

\subsection{結論}
% レポートの結論を記述

% 参考文献
\begin{thebibliography}{9}
\bibitem{ref1} 著者名, ``論文・書籍タイトル'', 出版社, 出版年.
\bibitem{ref2} 著者名, ``論文・書籍タイトル'', 出版社, 出版年.
\bibitem{ref3} 著者名, ``論文・書籍タイトル'', 出版社, 出版年.
\end{thebibliography}

% 付録(必要に応じて)
\appendix
\section{詳細仕様}
% 詳細な技術仕様等を記述

\section{コスト試算詳細}
% コスト試算の詳細を記述

\end{document}
